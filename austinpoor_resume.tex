%%%%%%%%%%%%%%%%%
% This is an example CV created using altacv.cls (v1.1.5, 1 December 2018) written by
% LianTze Lim (liantze@gmail.com), based on the
% Cv created by BusinessInsider at http://www.businessinsider.my/a-sample-resume-for-marissa-mayer-2016-7/?r=US&IR=T
%
%% It may be distributed and/or modified under the
%% conditions of the LaTeX Project Public License, either version 1.3
%% of this license or (at your option) any later version.
%% The latest version of this license is in
%%    http://www.latex-project.org/lppl.txt
%% and version 1.3 or later is part of all distributions of LaTeX
%% version 2003/12/01 or later.
%%%%%%%%%%%%%%%%

%% If you are using \orcid or academicons
%% icons, make sure you have the academicons
%% option here, and compile with XeLaTeX
%% or LuaLaTeX.
% \documentclass[10pt,a4paper,academicons]{altacv}

%% Use the "normalphoto" option if you want a normal photo instead of cropped to a circle
% \documentclass[10pt,a4paper,normalphoto]{altacv}

\documentclass[8pt,letter,ragged2e]{altacv}

%% AltaCV uses the fontawesome and academicon fonts
%% and packages.
%% See texdoc.net/pkg/fontawecome and http://texdoc.net/pkg/academicons for full list of symbols. You MUST compile with XeLaTeX or LuaLaTeX if you want to use academicons.

% Change the page layout if you need to
\geometry{left=1cm,right=7.0cm,marginparwidth=5.0cm,marginparsep=0.8cm,top=1.0cm,bottom=1.25cm}

% Change the font if you want to, depending on whether
% you're using pdflatex or xelatex/lualatex
%\ifxetexorluatex
%  % If using xelatex or lualatex:
%  \setmainfont{Carlito}
%\else
%  % If using pdflatex:
%  \usepackage[utf8]{inputenc}
%  \usepackage[T1]{fontenc}
%  \usepackage[default]{lato}
%\fi
\usepackage[utf8]{inputenc}
\usepackage[T1]{fontenc}
\usepackage[default]{lato}

\usepackage{multicol}

% Change the colours if you want to
\definecolor{VividPurple}{HTML}{3E0097}
\definecolor{SlateGrey}{HTML}{2E2E2E}
\definecolor{LightGrey}{HTML}{37474F}
\colorlet{heading}{VividPurple}
\colorlet{accent}{VividPurple}
\colorlet{emphasis}{SlateGrey}
\colorlet{body}{LightGrey}

% Change the bullets for itemize and rating marker
% for \cvskill if you want to
\renewcommand{\itemmarker}{{\small\textbullet}}
\renewcommand{\ratingmarker}{\faCircle}

%% sample.bib contains your publications
% \addbibresource{sample.bib}

\begin{document}
\name{Austin Poor}
% \tagline{Seeking a 6 month internship starting from March 2020}
% Cropped to square from https://en.wikipedia.org/wiki/Marissa_Mayer#/media/File:Marissa_Mayer_May_2014_(cropped).jpg, CC-BY 2.0
\photo{2.5cm}{ap-propic}
\personalinfo{%
  % Not all of these are required!
  % You can add your own with \printinfo{symbol}{detail}
    \email{austinpoor@gmail.com}
    \phone{(203) 558-4619}
    \location{Queens, NY}
    \linkedin{linkedin.com/in/austinpoor}
    \github{github.com/a-poor}
%   \orcid{orcid.org/0000-0000-0000-0000} % Obviously making this up too. If you want to use this field (and also other academicons symbols), add "academicons" option to \documentclass{altacv}
}

%% Make the header extend all the way to the right, if you want.
\begin{fullwidth}
\makecvheader
\end{fullwidth}

\vspace{-10px}

%% Depending on your tastes, you may want to make fonts of itemize environments slightly smaller
\AtBeginEnvironment{itemize}{\small}

%% Provide the file name containing the sidebar contents as an optional parameter to \cvsection.
%% You can always just use \marginpar{...} if you do
%% not need to align the top of the contents to any
%% \cvsection title in the "main" bar.

\cvsection[page1sidebar]{Experience}

\cvevent{Data Scientist}{Metis Data Science Bootcamp}{Jan 2020 -- March 2020}{New York, NY}

Highly selective, accredited 12-week immersive data science bootcamp focused on Python, statistical modeling, machine learning, visualization, and communication of results. See projects below.

\vspace{10px}

\cvevent{Assistant to the Executive Creative Director}{CHRLX}{Oct 2014 -- Aug 2017}{New York, NY}

Responsible for managing Executive Creative Director's priorities, schedule, and following employee task progress. On behalf of the ECD, coordinated with producers and technical staff on client project delivery assignments and issue resolution. Further, served as fill-in technical resource on multiple projects. Worked with clients such as Nike, Cinnamon Toast Crunch, Verizon FiOS, and Subway.
% \begin{itemize}
    % \item Coordinated with multiple producers across multiple projects
    % \item Part of the team that developed a VR app with the company's demo reel – responsible for building the Unity app for iOS
    % \item Junior Designer: Designed and produced PDF presentations of storyboards and style frames using Adobe suite
    % \item Junior Producer: Managed a small project for the CHRLX President – corresponded with multiple artists on-time as well as on-budget
    % \item Assistant Editor: Responsible for researching, collecting, and curating stock videos and stills for use by editors and compositors in the production process
    % \item Assistant Stage Manager for the CHRLX studio stage
% \end{itemize}

\vspace{10px}

\cvsection{Projects}

\cvevent{Save the Dinosaurs: Creating a Bot to Play the Chrome Dino Game}{}{}{}
Using deep reinforcement learning to train an agent to play the Chrome "No-Internet" Dinosaur game.
\begin{itemize}
    \item Used Selenium to control Chrome and test the bots' performance
    \item Created multiple "heuristic" bots to play based on hard-coded rules
    \item Used grid-search to optimize the heuristic bot's strategy
    \item Created a Deep Q-Learning bot using Tensorflow
    \item Got high score: 17,959 (about 5x the avg human high score)
\end{itemize}
Here's a link to the repo: github.com/a-poor/chrome-dino-solver
% \cvtag{MbedOS 5} 
% \cvtag{ARM}
% \cvtag{C/C++}
% \cvtag{Data Aquisition}
% \cvtag{NodeJs}
% \cvtag{Altium}
\vspace{2px}

\cvevent{AustinRecommendsMovies.com}{}{}{}
Built a Fask web app to recommend movies to users using collaborative filtering and content-based filtering.
\begin{itemize}
    \item Merged three large datasets (movie plot summaries, user reviews, movie metadata)
    \item Stored data using PostgreSQL and GCP
    \item Used NMF to create topic vectors based on film summaries
    \item Calculate user-user and film-film similarity using SQL
    \item Collaborative filtering: Recommend movies liked by similar users
    \item Content-based filtering: Recommend movies by plot-vector similarity
    \item User stats: Used Bokeh to plot user-rating distributions
\end{itemize}
Here's a link to the repo: github.com/a-poor/movie-recs
\vspace{2px}

\cvevent{Spotify Skip Prediction}{}{}{}
Analyzed Spotify data on user listening sessions to predict the likelihood of a user skipping a song.
\begin{itemize}
    \item Stored the large dataset in PostgreSQL on AWS
    \item Used feature engineering to account for sequential data
    \item Performed classification with multiple model types
    \item LightGBM Classifier got a final accuracy of 0.73
\end{itemize}
Read more about it here: https://towardsdatascience.com/predicting-spotify-track-skips-49cf4a48b2a5
And here's a link to the repo: github.com/a-poor/spotify-skip-prediction
\vspace{2px}









\cvsection{Education}

\cvevent{BA – Computer Science}{Sarah Lawrence College}{Sept 2017 - Dec 2019}{Bronxville, NY}
Studied computer science with a focus on data science. Each course involved a semester-long, in-depth project related to the course. Select courses:
\vspace{-10px}
\begin{multicols}{3}
    \begin{itemize}
        \item Bio-Inspired Artificial Intelligence
        \item Databases
        \item Computer Organization.
    \end{itemize}
\end{multicols}
\vspace{-10px}

Select project topics:
\begin{itemize}
    \item Flask App for College Student Course Sign-Up
    \item Predicting the Political Leaning of News Articles with Deep Learning
    \item Used data science tools to investigated litigation practices in NYC housing court – focusing on improper service of process
\end{itemize}

\cvevent{}{University of Connecticut}{Sept 2012 - June 2014}{Storrs, CT}
Took courses towards a BA in Communications

\clearpage

\end{document}
